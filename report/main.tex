% -------------------------------- SVILUPPO -------------------------------------

% ---------------- RELAZIONE PROGETTO DI PROGRAMMAZIONE AD OGGETTI (OOP) --------
\documentclass[a4paper,12pt]{report}

% ----------------------------- PREAMBLE --------------------------------------- 

\usepackage{lmodern}
\usepackage{alltt, fancyvrb, url}
\usepackage{float}
\usepackage{graphicx}
\usepackage[utf8]{inputenc}
\usepackage{hyperref}
\usepackage{amsmath,amssymb,amsthm}

\usepackage[italian]{babel}

\usepackage[italian]{cleveref}

\usepackage{comment}
\usepackage{microtype}
\usepackage{fancyhdr}

\usepackage[scaled=.92]{helvet}
\usepackage[T1]{fontenc}

\usepackage{lscape}

% hyperref settings
\hypersetup{
	colorlinks=true,
	linkcolor=black, %blue
	filecolor=magenta,      
	urlcolor=cyan,
	pdftitle={Sharelatex Example},
	bookmarks=true,
	pdfpagemode=FullScreen,
}

% ----------------------------- PREAMBLE END -----------------------------------

\makeindex

\title{\textbf{Progetto di tecnologie web report}}
\author{Luca Rengo, Alessandro Pioggia}

\begin{document}
	
	\makeatletter
	\begin{titlepage}
		\begin{center}
			{\Huge  \@title }\\[3ex] 
			{\large  \@author}\\[3ex] 
			{\large \@date}
		\end{center}
	\end{titlepage}
	\makeatother
	\thispagestyle{empty}
	\newpage
	
	%\maketitle
	
	\tableofcontents

	\newpage
	
	% \input: import the commands from filename.tex to target file.
	
	% \include: does a \clearpage and does an \input.
	\section{Dominio applicativo}
	Vendita di panini con consegna a domicilio in 8 minuti max, eco friendly(consegne in bici elettrica e tesla).
	\section{Tecnologie utilizzate}
	\begin{itemize}
		\item css : 
		\item lato server : 
		\item lato client : 
	\end{itemize}
	\section{Applicativo}
	\subsection{Versione 1 - base(solo componenti obbligatorie)}
	\begin{itemize}
		\item login e pagina dedicata a venditore e cliente
		\item aggiungere prodotti lato venditore
		\item creare una lista di prodotti(paninozzi) acquistabili
		\item carrello e acquisto di prodotti selezionati da lista
		\item mostrare i prodotti nel carrello(tabella dinamica)
		\item mostrare lista dei prodotti, lato venditore(tabella dinamica)
		\item notifica per l'acquisto(successo o insuccesso)
		\item notifica per la vendita
		\item simulazione del pagamento
		\item in caso di portafogli vuoto warning
	\end{itemize}
	\subsection{Versione definitiva - con effetti wow}
	\begin{itemize}
		\item salatura password
		\item utilizzo di ajax
		\item accessibile per chi utilizza lo screen reader
		\item aggiunta di recensioni 
	\end{itemize}
	\section{Software per Mockup utilizzati}
	\begin{itemize}
		\item figma
		\item balsamiq
	\end{itemize}
	\section{Guida utente}
	L'utente in prima istanza visualizzerà la homepage del sito, da lì avrà la possibilità di accedere.
	L'accesso potrà essere sia customer-side(cliente) e vendor-side(venditore).
	Dal lato customer-side potrà comprare e farsi spedire i prodotti, lasciare recensioni, visualizzare il catalogo, ecc. ecc.
	Dal lato vendor-side verranno gestite le liste di prodotti, ovviamente la vendita e tanto altro.
\end{document}
